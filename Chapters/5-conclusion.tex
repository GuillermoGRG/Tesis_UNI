% ============
% Conclusiones
% ============

Los modelos $3-3-1$ nos permiten explicar fenómenos como la gran cantidad de una materia no luminosa, llamada materia oscura, la cual es dominante en el universo. Esta explicación escapa al modelo estándar, pero gracias a su gran contenido de nuevas partículas e interacciones, los modelos como el $\nu_R-331$, lo pueden hacer por medio de uno o más candidatos a materia oscura.

En este trabajo hemos calculado la densidad de reliquia de un candidato a materia oscura en el modelo $\nu_R-331$ dado por un escalar complejo $\eta_3^0$ que acopla con otros escalares del modelo ($h,H,H_3,H^{\pm},\eta^{\pm}$), los bosones vectoriales ($Z,Z',Y_1,Y_2,W'$), los quarks conocidos $u$, $s$ y $t$, asi como los nuevos quarks super pesados del modelo ($Q$).

Vimos que hay tres procesos principales que determinan la abundancia de la materia oscura en este caso: $\mathcal{DM}\, \mathcal{DM} \to \phi \phi$, $\mathcal{DM}\, \mathcal{DM} \to \phi^* \to f\bar{f}$ y $\mathcal{DM}\, \mathcal{DM} \to q\bar{q}$. Hemos trabajado con cada uno de ellos y hemos encontrado regiones del espacio de parámetros donde estamos de acuerdo con las observaciones experimentales para la densidad de materia oscura. En particular, hemos trabajado en el caso que la materia oscura sea más ligera que los quarks super pesados del modelo para asi asegurar su estabilidad. Es posible relajar esta hipótesis ya que solo requerimos que la materia oscura tenga un tiempo de vida medio mayor a la edad del universo.

Trabajos futuros pueden ser hechos en base a esta tesis. Por ejemplo, el estudio de detección directa e indirecta de este candidato a materia oscura requiere analizar resultados experimentales de las colaboraciones XENONnT, LUX, PandaX y DarkSide-20k y Fermi-LAT, H.E.S.S, MAGIC, VERITAS, los cuales cubren masas de materia oscura en el orden de cientos de GeV a decenas de TeV, y futuros experimentos como CTA.

%El estudio de la estabilidad del vacío \cite{sanchez2019vacuum}

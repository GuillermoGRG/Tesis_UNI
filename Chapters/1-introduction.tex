Comprender el universo donde vivimos siempre ha sido de gran interés para la humanidad. Cerca de 200 a.C., el sabio griego Eratóstenes de Cirene pudo determinar la longitud de la circunferencia de la Tierra al contratar un profesional que midió la distancia entre Siena y Alejandría y hallar el ángulo de inclinación de la sombra de un palo fijado en el suelo en Alejandría al medio dia el dia del solsticio de verano cuando la sombra era nula en Siena.

En esa época en la antigua Grecia ya existía la teoría filosófica del Atomismo según la cual pequeñas partículas indivisibles, llamadas átomos, componen el universo. Sin embargo, no es hasta finales del sigo XVIII que John Dalton propuso una teoría atómica moderna donde los átomos se usan para explicar los diferentes elementos químicos y sus reacciones. En esta teoría los diferentes tipos de átomos seguían siendo considerados indivisibles.

Más adelante, a finales del siglo XIX, la teoría atómica cambiaría con los experimentos de J. J. Thomson quien propuso que los átomos tienen estructura. Así, creó su modelo atómico donde el átomo es una esfera cargada positivamente y en ella se encuentran distribuidos los electrones de manera tal que el conjunto total sea eléctricamente nulo. 

Al descubrir el núcleo atómico, Ernest Rutherford creó un nuevo modelo para el átomo donde el núcleo era una pequeña región densa cargada positivamente y los electrones orbitan alrededor de ella. El modelo mejoró posteriormente con el descubrimiento de los neutrones por parte de James Chadwick. Actualmente se entiende que el núcleo atómico está formado por protones y neutrones.

Con la invención de la teoría de la relatividad especial de Einstein, la mecánica cuántica  y su unión en la teoría cuántica de campos a inicios del siglo XX, se pudieron sentar las bases matemáticas rigurosas para entender mejor la estructura del átomo y sus interacciones. Se entendió que existen tres tipos de interacciones cuánticas: la electrodinámica que mantiene a los electrones `alrededor' del núcleo, la interacción fuerte que mantiene al núcleo unido a si mismo y la interacción débil que permite la transformación del núcleo en uno de otro tipo.

A finales de los años 1970 se había construido una teoría que era capaz de explicar todos los experimentos en física de partículas a la fecha: el modelo estándar de la física de partículas. Aquí, los protones y neutrones se entienden como compuestos de quarks unidos por la `fuerza' nuclear fuerte que acabamos de mencionar. 


La última pieza del `rompecabezas' del modelo estándar era el bosón de Higgs que fue propuesto en 1964 y descubierto en el CERN en 2012 por las colaboraciones ATLAS \cite{ATLAS:2012yve} y CMS \cite{CMS:2012qbp}. El bosón de Higgs es la partícula del campo de Higgs y tiene carga eléctrica neutra, tiempo de vida corto y spin cero. Sus interacciones con el leptón tau y el quark top fueron medidas en 2016 y 2018, respectivamente. Esto último es interesante porque el tau y el top son los fermiones más pesados del modelo estándar y por eso son los que interactúan más fuertemente con el campo de Higgs: \textit{mientras más fuerte sea la interacción con el campo de Higgs, mayor será la masa de dicha partícula}.

Existe, sin embargo, una partícula que aparentemente no interacciona con el campo de Higgs. Esta partícula es el neutrino, el cual tiene masa nula en la teoría del modelo estándar. Esta partícula viene en tres `sabores': electrónico, muónico y tauónico, y tienen la asombrosa propiedad de oscilar entre esos sabores. Este fenómeno llamado de \textit{oscilación de neutrinos} es posible solo cuando los neutrinos tienen masa y estas son diferentes una de la otra. 

Esta aparente contradicción con el mecanismo de generación de masa por medio de la interacción con el campo de Higgs puede desaparecer postulando nuevas partículas que acoplan con los neutrinos y el Higgs. Estas partículas tendrían que ser muy pesadas en comparación con las energías de los experimentos terrestres ya que no han sido detectadas. Se puede decir que, de existir, estas partículas pertenecerían a un sector oscuro, diferente del modelo estándar.

La idea de un sector oscuro no es nueva. A inicios del siglo XX, observaciones astronómicas permitieron concluir que hay un tipo de materia no lumínica la cual no se ha podido detectar directamente pero su influencia gravitacional está presente (y es dominante) en las galáxias y los cúmulos de galáxias. Es así como nace la propuesta de \textit{materia oscura}, un nuevo (o nuevos) tipo(s) de materia que no es descrita por partículas del modelo estándar.

En este trabajo estudiaremos la materia oscura en una extensión del modelo estándar llamada \textit{modelos} $3-3-1$ \cite{PhysRevD.46.410}. Estos modelos son ricos en nuevas partículas e interacciones, lo que es deseable para hacer fenomenología de altas energías. Estos modelos han sido exitosos en, por ejemplo, explicar que el número de familias fermiónicas sea igual a tres \cite{Foot:1992rh} y la cuantización de la carga eléctrica \cite{deSousaPires:1998jc}. En particular, nos enfocaremos en el modelo $\nu_R$-331 \cite{Singer:1980sw} y calcularemos la densidad de relíquia para un candidato a materia oscura dada por un escalar complejo con masa desde los cientos de GeV hasta decenas de TeV. 

%En este trabajo analizamos límites experimentales de detección indirecta de materia oscura para un escalar complejo proveniente de una extensión del modelo estándar llamada \textit{modelos} $3-3-1$. En estos modelos se postula la existencia de más de un campo escalar `tipo Higgs', los cuales son necesários para reproducir los resultados del modelo estándar a `bajas energías' con unas pequeñas correcciones. En particular, nos enfocaremos en el modelo $3-3-1$ Económico y usaremos límites experimentales obtenidos por la colaboración H.E.S.S. que investiga rayos gamma usando telescópios atmosféricos de tipo Cherenkov.  


\color{black}
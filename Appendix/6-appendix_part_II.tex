%\section[\hspace{-0.14in}Matrices de masa en el $\nu_R-331$]{Matrices de masa en el $\nu_R-331$}
\section*{Matrices de masa en el $\nu_R-331$}

Sea la derivada covariante,
\begin{equation}
D_\mu \equiv \partial_\mu + ig A_\mu^a T^a + i g_X X B_\mu, 
\end{equation}
donde $T^a$ ($a=1-8$) son las matrices de Gell-Mann con normalización tr($T^aT^b$)=$\delta_{\rm ab}/2$ y $X$ es la carga $U(1)_X$ de $\rho, \eta$ y $\chi$, calculamos la relación entre los autoestados de masa $\{ A_{\mu},Z_{\mu}, Z'_{\mu},W_{\mu}^{\pm}, W'^{\pm}_{\mu},Y_{1\mu},Y_{2\mu} \}$ y los de simetría $\{B_{\mu}, A^a_{\mu}\}$,
\begin{equation}
(A_{\mu},Z_{1\mu}, Z_{2\mu}) = 
\left(
\begin{array}{ccccc}
\frac{\sqrt{3}}{2} \sin \theta_{\rm 331} & -\frac{1}{2} & \frac{\sqrt{3}}{2} \cos \theta_{\rm 331} \\
 -\frac{1}{2}\sin \theta_{\rm 331} & -\frac{\sqrt{3}}{2} & -\frac{1}{2} \cos \theta_{\rm 331} \\
 \cos \theta_{\rm 331} & 0 & -\sin \theta_{\rm 331} 
\end{array}
\right)
\left(
\begin{array}{c}
A_{3\mu} \\
A_{8\mu} \\
B_{\mu}  
\end{array}
\right),
\end{equation}
donde $\theta_{331}$ está definido por $\tan \theta_{331} = \frac{2g_X}{\sqrt{3}g}$ y además
\begin{equation}
\left(Z_{\mu},Z'_{\mu}\right) =
\left(
\begin{array}{cc}
 c_a & s_a \\
 -s_a & c_a 
\end{array}
\right)
\left(
\begin{array}{c}
 Z_{1\mu}  \\
 Z_{2\mu}  
\end{array}
\right).
\end{equation}
Aqui $c_a\equiv \cos a$, $s_a\equiv \sin a$\footnote{Cuando $v_{\rm SM} \ll v_\chi$, tenemos $s_a \approx \frac{1}{2c_W}$.} y
\begin{equation}
\tan 2a = \frac{2\sqrt{3}\cos \theta_{331}(v_\chi^2-s_\beta^2 v_{\rm SM}^2)}{-4(v_\chi^2 + v_{\rm SM}^2) + 3 (v_\chi^2 + s_\beta^2v_{\rm SM}^2)(1+\cos^2 \theta_{331})}.
\label{tan2a}
\end{equation}

Para los otros bosones de gauge tenemos
\begin{equation}
Y_{1\mu} = A_{4\mu}, Y_{2\mu} = A_{5\mu}, W_\mu^\pm = \frac{1}{\sqrt{2}} \left(  A_{1\mu} \mp iA_{2\mu} \right), W'^\pm_\mu = \frac{1}{\sqrt{2}} \left(  A_{6\mu} \mp iA_{7\mu} \right)  
\end{equation}

La matriz de masa para los bosones vectoriales en la base $\{A_\mu,Z_{1\mu},Z_{2\mu},Y_{1\mu},Y_{2\mu}\}$, donde $A_\mu$ es el fotón, es
\begin{equation}
\mathcal{M}^2 = \frac{g^2}{4}
\left(
\begin{array}{ccccc}
 0 & 0 & 0 & 0 & 0 \\
 0 & {v_{\rm SM}}^2 s_{\beta }^2+{v_\chi }^2 & \frac{{v_{\rm SM}}^2 s_{\beta }^2-{v_\chi }^2}{\sqrt{3} c_{331}} & 0 & 0 \\
 0 & \frac{{v_{\rm SM}}^2 s_{\beta }^2-{v_\chi }^2}{\sqrt{3} c_{331}} & \frac{(3 c_{\beta }^2+1){v_{\rm SM}}^2+{v_\chi }^2}{3 c_{331}^2} & 0 & 0 \\
 0 & 0 & 0 & {v_{\rm SM}}^2 c_{\beta }^2+{v_\chi }^2 & 0 \\
 0 & 0 & 0 & 0 & {v_{\rm SM}}^2 c_{\beta }^2+{v_\chi }^2 \\
\end{array}
\right).
\end{equation}

\  \ 

Para los escalares pares por CP, la matriz de masa es $\mathcal{M}_{p}^2$ en la base $\{ \text{Re} (\rho_2^0), \text{Re} (\eta_1^0), \text{Re} (\chi_3^0)$, $ \text{Re} (\chi_1^0), \text{Re} (\eta_3^0) \}$ es:
\begin{small}
\begin{equation}
\mathcal{M}_{p}^2 =\frac{1}{4}f\,  v_\chi
\left(
\begin{array}{ccccc}
 \frac{{v_\eta }}{{v_\rho }}+\frac{4 {\lambda_ 1} {v_\rho }^2}{{f} {v_\chi }} & \frac{2 {\lambda_ 6} {v_\eta } {v_\rho }}{{f} {v_\chi }}-1 & \frac{2 {\lambda_ 4} {v_\rho }}{{f}}-\frac{{v_\eta }}{{v_\chi }} & 0 & 0 \\
 \frac{2 {\lambda_ 6} {v_\eta } {v_\rho }}{{f} {v_\chi }}-1 & \frac{4 {\lambda_ 2} {v_\eta }^2}{{f} {v_\chi }}+\frac{{v_\rho }}{{v_\eta }} & \frac{2 {\lambda_ 5} {v_\eta }}{{f}}-\frac{{v_\rho }}{{v_\chi }} & 0 & 0 \\
 \frac{2 {\lambda_ 4} {v_\rho }}{{f}}-\frac{{v_\eta }}{{v_\chi }} & \frac{2 {\lambda_ 5} {v_\eta }}{{f}}-\frac{{v_\rho }}{{v_\chi }} & \frac{{v_\eta } {v_\rho }}{{v_\chi }^2}+\frac{4 {\lambda_ 3} {v_\chi }}{{f}} & 0 & 0 \\
 0 & 0 & 0 & \frac{{v_\eta } ({v_\eta } {v_\chi } (2 {\lambda_ {10}}+{\lambda_ 8})+{f} {v_\rho })}{{f} {v_\chi }^2} & \frac{{v_\eta } (2 {\lambda_ {10}}+{\lambda_ 8})}{{f}}+\frac{{v_\rho }}{{v_\chi }} \\
 0 & 0 & 0 & \frac{{v_\eta } (2 {\lambda_ {10}}+{\lambda_ 8})}{{f}}+\frac{{v_\rho }}{{v_\chi }} & \frac{{v_\rho }}{{v_\eta }}+\frac{{v_\chi } (2 {\lambda_ {10}}+{\lambda_ 8})}{{f}} \\
\end{array}
\right)
\end{equation}
\end{small}

Para simplificar los cálculos y obtener expresiones analíticas útiles, se define el límite de alineamiento para $\mathcal{M}_p^2$ al hacer \cite{Okada:2016whh},
\begin{equation}
\frac{2 {\lambda_ 4} {v_\rho }}{{f}} -\frac{{v_\eta }}{{v_\chi }} =0, \hspace{0.2in} \frac{{v_\rho }}{{v_\chi }} - \frac{2 {\lambda_ 5} {v_\eta }}{{f}} =0,
\label{alignment}
\end{equation}
donde $\mathcal{M}_{p}^2$ se vuelve una matriz diagonal por bloques. Resolviendo estas ecuaciones para $\lambda_4$ y $\lambda_5$ obtenemos,
\begin{equation}
\lambda_4 = \frac{t_\beta}{2} \frac{f}{v_\chi}, \hspace{0.25in} \lambda_5 = \frac{1}{2t_\beta} \frac{f}{v_\chi},
\end{equation}
y, como $\lambda_4$ y $\lambda_5 < 4\pi$, se obtienen límites para $\beta, f$ y $v_\chi$. Sin embargo, valores para $\beta$ cerca de cero o de $\pi/2$ no son interesantes por ser ajustes finos y hacen que las masas de $H$ y $A$ diverjan. En este trabajo evitamos estudiar estos casos y, por lo tanto, en el límite $f\ll v_\chi$ tenemos que ambos $\lambda's$ son positivos y muy cercanos a cero, quedando lejos de sus límites superiores y manteniendo la teoría en la región perturbativo.

Como resultado, obtenemos
\begin{eqnarray}
m_h^2 &=& \frac{1}{2}v_{\rm SM}^2 \left(a+b-\sqrt{(a-b)^2+c^2} \right) \nonumber \\
m_H^2 &=& \frac{1}{2}v_{\rm SM}^2 \left(a+b+\sqrt{(a-b)^2+c^2} \right) \nonumber \\
m_{H_3}^2 &=& 2\lambda_3 v_\chi^2 + \frac{f}{4v_\chi}s_{2\beta}  v_{\rm SM}^2 \nonumber \\
m_{\eta_R}^2 &=& \frac{v_\eta^2 + v_\chi^2}{2v_\eta v_\chi} \bigg( 2v_\eta v_\chi \lambda_{10} + f\,v_\rho + v_\eta v_\chi \lambda_8 \bigg)
\end{eqnarray}
donde $a= 2\lambda_1 c_\beta^2+ \frac{t_\beta}{2}  \frac{f \, v_\chi}{v_{\rm SM}^2}$, $b=2\lambda_2 s_\beta^2+ \frac{1}{2t_\beta} \frac{f \, v_\chi}{v_{\rm SM}^2}$ y $c=  \lambda_6 s_{2\beta}-  \frac{f \, v_\chi}{v_{\rm SM}^2}$. Los valores propios\footnote{Los valores propios $h$ y $H$ son muy grandes para escribirse aqui, pero se ha encontrado aproximaciones para ellos en el caso $f\, v_\chi = v_{\rm SM}^2$, el cual es mostrado en el cuadro \ref{EigenVals331}.} son (sin normalizar)
\begin{eqnarray}
H_3 &=& \text{Re}(\chi_3^0), \nonumber \\
\eta_R &=& \frac{v_\eta}{v_\chi} \text{Re}(\chi_1^0) + \text{Re}(\eta_3^0).
\end{eqnarray}

\  \ 

Para los escalares impares por CP, la matriz de masa es $\mathcal{M}_{i}^2$ en la base $\{ \text{Im}(\rho_2^0), \text{Im}(\eta_1^0)$, $\text{Im}(\chi_3^0), \text{Im}(\chi_1^0), \text{Im}(\eta_3^0) \}$  tenemos:
\begin{small}
\[
\mathcal{M}_{i}^2 =\frac{1}{4} f\, v_\chi
\left(
\begin{array}{ccccc}
 \frac{{v_\eta }}{{v_\rho }} & 1 & \frac{{v_\eta }}{{v_\chi }} & 0 & 0 \\
 1 & \frac{{v_\rho }}{{v_\eta }} & \frac{{v_\rho }}{{v_\chi }} & 0 & 0 \\
 \frac{{v_\eta }}{{v_\chi }} & \frac{{v_\rho }}{{v_\chi }} & \frac{{v_\eta } {v_\rho }}{{v_\chi }^2} & 0 & 0 \\
 0 & 0 & 0 & \frac{{v_\eta } ({v_\eta } {v_\chi } ({\lambda_ 8}-2 {\lambda_ {10}})+{f} {v_\rho })}{{f} {v_\chi }^2} & \frac{{v_\eta } (2 {\lambda_ {10}}-{\lambda_ 8})}{{f}}-\frac{{v_\rho }}{{v_\chi }} \\
 0 & 0 & 0 & \frac{{v_\eta } (2 {\lambda_ {10}}-{\lambda_ 8})}{{f}}-\frac{{v_\rho }}{{v_\chi }} & \frac{{v_\rho }}{{v_\eta }}+\frac{{v_\chi } ({\lambda_ 8}-2 {\lambda_ {10}})}{{f}} \\
\end{array}
\right)
\]
\end{small}

Aqui obtenemos
\begin{eqnarray}
m_A^2 &=& \frac{v_\rho v_\chi}{2v_\eta} f + \frac{v_\eta (v_\rho^2 + v_\chi^2)}{2v_\rho v_\chi}f, \nonumber \\
m_{\eta_I}^2 &=& \frac{v_\eta^2 + v_\chi^2}{2v_\eta v_\chi} \bigg( v_\rho f+v_\eta v_\chi (\lambda_8-\lambda_{10}) \bigg),
\end{eqnarray}
donde (sin normalizar)
\begin{eqnarray}
A &=& \frac{v_\chi}{v_\rho} \text{Im}(\rho_2^0) + \frac{v_\chi}{v_\eta} \text{Im}(\eta_1^0) + \text{Im}(\chi_3^0), \nonumber \\
\eta_I &=& -\frac{v_\eta}{v_\chi} \text{Im}(\chi_1^0) + \text{Im}(\eta_3^0).
\end{eqnarray}

